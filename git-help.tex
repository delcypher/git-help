\documentclass[a4paper,10pt,fleqn]{article}
\usepackage{fancyhdr}
\usepackage{amsmath}
\usepackage[vmarginratio=1:1,vscale=0.85,left=1.5cm,right=1.5cm]{geometry}
\usepackage[usenames,dvipsnames]{color}
\usepackage{ulem} %used for strikeout
\usepackage[pdftex,colorlinks]{hyperref}
\usepackage{graphicx}
\usepackage{subfig}

\usepackage{listings}

\newcommand{\danspagestyle}
{
%blank header & footer
\fancyhf{}
%set footer and header values
%\lhead{\textsf{Imperial College London}}
\rhead{}
\cfoot{page \thepage}
\rfoot{Daniel Liew}
}
\pagestyle{fancy}
\danspagestyle
%remove header ruler
\renewcommand{\headrulewidth}{0pt}
%set gap between paragraphs
\setlength{\parskip}{1ex plus0.5ex minus0.1ex}

%modify plain style so first pages of chapter get correct style.
\fancypagestyle{plain}{
\danspagestyle
}


%opening
\title{Getting Started with \texttt{git}}
\author{Daniel Liew}

%set code style
\lstset{language=bash,basicstyle=\footnotesize\ttfamily,breaklines=true,commentstyle=\color{blue},frame=single,showstringspaces=false}

\begin{document}
\maketitle

\section{Introduction}
You probably noticed the e-mail from \texttt{CSG} telling you about your group folder. They also talked about \href{http://en.wikipedia.org/wiki/Version_control}{version control}. They gave a quick start guide for \href{http://subversion.apache.org/}{\texttt{svn}} but not for \href{http://git-scm.com/}{\texttt{git}} so I thought I'd do one. Note that I'm quite biased, I've used \texttt{git} a lot more than I've used \texttt{svn}. Text in a colour that isn't black in this document are links.

\subsection{Why use \texttt{git} over \texttt{svn}}
\texttt{svn} is a centralised version control system. This means that all the data regarding your project is kept on a central server. Here are a few bad things about it
	\begin{itemize}
	\item If the server goes down or you loose your network connection you can't really work, you can edit/view files but that's about it. So you can't do important things like look at project history, change branches and commit your changes!
	\item If you loose the data on the server you are screwed because that was the only copy of the project history, branches.
	\item Branching is complicated compared to \texttt{git} (I don't know this one for sure I just read it was the case).
	\item Svn does not cryptographically secure the files it tracks.
	\end{itemize}

\texttt{git} is a distributed version control system which makes it very flexible. Here are a few great things about it
	\begin{itemize}
		\item Unlike \texttt{svn} when you work with projects you get the whole thing (branches, history, etc.) so you do not need network access to work.
		\item If you loose the repository on some central server it probably won't matter because everyone who used the repository recently has a complete copy of it which can be very easily be used to set up a new repository.
		\item Branching is relatively simple.
		\item Git cryptographically secures the files it tracks making it very difficult to tamper with files without someone noticing.
		\item Being decentralised allows different work flows. You can a central repository just like you would with \texttt{svn} but you can also take select changes from different people's repositories.
	\end{itemize}
	
	Don't take my word for it though, Linus Torvalds the original author of \texttt{git} gave a talk about this subject \url{http://www.youtube.com/watch?v=4XpnKHJAok8}.

\section{Installing \texttt{git}}
	\begin{description}
	 \item[Windows] Use \url{http://code.google.com/p/msysgit/}
	 \item[Mac OSX] Use \url{http://code.google.com/p/git-osx-installer/} or macports \url{http://www.macports.org/}
	 \item[Linux] It is already installed on the DoC computers but if you wish to install on your own machine you should find it in your distribution's package repositories.	
	\end{description}


\section{Setting up a centralised \texttt{git} repository for your group}
\texttt{git} is primarily a command line based tool but there quite a few GUI front ends for it as well as plug-in for IDEs such as Eclipse. I'll concentrate on command line usage. Note even though I talk about ``centralised repositories'' when you do a ``clone'' you get an entire copy of the repository and not just the files (like \texttt{svn}). A centralised repository suggests a centralised workflow but it is not the only workflow you can have but for a small group who are new to \texttt{git} it is probably the simplest thing to do.

		\subsection{Public centralised repository}
		If your supervisor doesn't care that your code is open source then an easy option is to use a public repository hosting website for your code. These are free provided your code is open source. Here are two hosting sites I'm aware of. They both have good documentation so I won't cover how to use them.
		\begin{itemize}
			\item \url{http://www.github.com} . It has cool features like wikis and issue trackers as well as a very pretty interface for viewing repositories. Here's an example repository (one of mine) \url{https://github.com/delcypher/git-help} which is the source code to make this PDF document.
			\item \url{http://gitorious.org/} . I've not used it so I can't tell you how good it is. Here's an example repository \url{http://qt.gitorious.org/qt/qt}.
		\end{itemize}

		\subsection{Private centralised repository}
		If your supervisor doesn't want the code publicly available or you want to have a repository for anything you don't think should be public then this is for you.

		This repository is going act as a central repository which everyone in your group ``pushes'' and ``pulls'' from. 

		You can host your project pretty much anywhere you want in principle but it'll be a lot more useful if it's accessible from outside Imperial. A good way of doing this is creating a repository in your group folder (you can also create one in one of your group member's home directory). It can then be accessed directly internal to Imperial and via the SSH protocol from outside Imperial.

		I will assume that we wish to create a repository in the imaginary folder \texttt{/vol/project/2011/530/some-group/}, replace as appropriate. Lines beginning with a \texttt{\$} symbol indicate commands you should type in (do not type the \texttt{\$}).

		\begin{enumerate}
		\item First we're going to set up your identity. Everyone in your group needs to do this as this identity will be used against changes that are made by that user. Run the following from any directory. Replace ``Your Name'' and ``username@doc.ic.ac.uk'' as appropriate. The ``core.editor'' variable sets which text editor git will use for commit messages. If you prefer \texttt{vim} then use \texttt{/usr/bin/vim}
			\begin{lstlisting}
$ git config --global user.name "Your Name"
$ git config --global user.email "username@doc.ic.ac.uk"
$ git config --global core.editor /usr/bin/emacs
		\end{lstlisting}

		You can confirm what you have set by running
			\begin{lstlisting}
$ git config --global --list
user.name=Your Name
user.email=username@imperial.ac.uk
core.editor /usr/bin/emacs
		\end{lstlisting}
		
		\item Now change to the directory you want to put your repository in
		\begin{lstlisting}
$ cd /vol/project/2011/530/some-group/
		\end{lstlisting}

		\item Now we will create an empty repository where ``repo-name'' should be a name of your choices (avoid spaces in the name, it makes things easier)
		\begin{lstlisting}
$ git init --bare --shared=group repo-name
Initialized empty shared Git repository in /vol/project/2011/530/some-group/git-test/
		\end{lstlisting}
		The \texttt{--bare} option creates a repository which allows other people to ``push'' there changes. The \texttt{--shared=group} option should create the repository with the correct permissions.

		\item Now we will check that the permissions on the directory are correct.
		\begin{lstlisting}[caption={Checking permissions},label={lst:perm}]
$ ls -ld repo-name/
total 0
drwxrwsr-x 7 dsl11 g1153011 111 2011-12-19 21:23 repo-name
		\end{lstlisting}

		The permissions mask (as in listing \ref{lst:perm}) which should be \texttt{drwxrwsr-x}. If this is not the case then run.
		\begin{lstlisting}
$ find repo-name/ -type d -exec chmod u=rwx,g=rws,o=rx \{} \;
$ find repo-name/ -type f -exec chmod u=rw,g=rw,o=r \{} \;
		\end{lstlisting}

		The group (as in listing \ref{lst:perm}) should be set to your group (in the example \texttt{g1153011}). To find out group name run
		\begin{lstlisting}
$ groups 
mcs doc-pgtaught g1153011
		\end{lstlisting}
		This will list the groups you are part of. Your group name will begin with ``g11''.

		If the group in listing \ref{lst:perm} does not match your group name (e.g. it is set to \texttt{mcs} instead) then run the following
		\begin{lstlisting}
$ chown -R :g1153011 repo-name/
		\end{lstlisting}

		\item Now we will ``clone'' the empty repository into your home directory and put an initial commit in with a small test file. Please note ``clone'' is an operation you only do once per repository. Once you've cloned you update it by ``pushing'' or ``pulling'' from other repositories.
		\begin{lstlisting}
$ cd ~
$ git clone  /vol/project/2011/530/some-group/repo-name/
Cloning into repo-name...
done.
warning: You appear to have cloned an empty repository.
$ cd repo-name/
$ touch readme
$ git add readme
$ git commit -m "Initial empty commit"
[master (root-commit) 1b95784] Initial empty commit
0 files changed, 0 insertions(+), 0 deletions(-)
create mode 100644 readme
		\end{lstlisting}

		\item Now we'll push this change to the repository in your group project so everyone can see it.
		\begin{lstlisting}
$ git push origin master
Counting objects: 3, done.
Writing objects: 100% (3/3), 220 bytes, done.
Total 3 (delta 0), reused 0 (delta 0)
Unpacking objects: 100% (3/3), done.
To /vol/project/2011/530/g1153011/repo-name
* [new branch]      master -> master
		\end{lstlisting}

		Now anyone in your group can clone this repository. There are two ways to do this
		\begin{itemize}
			\item If you want to clone from inside the DoC run in the directory you want your copy of the repository to be.
			\begin{lstlisting}
$ git clone /vol/project/2011/530/some-group/repo-name/
			\end{lstlisting}

			\item If you you want to access from outside the DoC (e.g. your home computer) run in the directory you want your copy of the repository to be. Note \texttt{username} is your username and \texttt{/vol/project/2011/530/some-group/repo-name/} is the path to your repository.
			\begin{lstlisting}
$ git clone username@shell1.doc.ic.ac.uk:/vol/project/2011/530/some-group/repo-name/
			\end{lstlisting}

		\end{itemize}

		\end{enumerate}

\section{Learning how to use \texttt{git}}
There are a lot of good resources for learning \texttt{git} out there. Here are a few of the best that I've found.
\begin{itemize}
	\item Progit is freely available book which I learnt \texttt{git} from last year. The first two chapters are the most important. \url{http://progit.org/book/} 
	\item Git community book. \url{http://book.git-scm.com/}.
	\item Documentation from the official website \url{http://git-scm.com/documentation}.
	\item There is in built git documentation which I use a lot. Just type on the command line
	\begin{lstlisting}
$ man git <command>
	\end{lstlisting}
	where \texttt{<command>} is the command you want to know about.
	\item Google. If you have a question about how to do something it \texttt{git} chances are someone has already asked about it.

\end{itemize}

\section{E-mails on push}
Cristian Cadar recommened having e-mails sent to the group everytime someone did a commit for \texttt{svn}. We can do something similar for \texttt{git} except that it doesn't make sense for e-mails to be sent when you do a commit because that commit is local to your repository so no one else can see it apart from you. Instead we setup your central repository (assuming you have one) to send out an e-mail everytime somone pushes into it. How this done depends on your setup please choose the relevant sub section for you.

\subsection{Private centralised repository}
We will add a ``post-receive hook'' to your central repository. This is just a bash shell script that is executed everytime someone pushes into that repository. It relies on the \texttt{sendmail} program to send mail as long as that is configured correct then everything should work. On the lab machines this already configured correctly.

First \texttt{cd} into your central repository folder, you should see the following folder structure.
\begin{lstlisting}
$ cd /vol/project/2011/530/some-group/repo-name/
$ ls -l
total 36
drwxr-xr-x  2 dan users 4096 Jan 12 09:42 branches
-rw-r--r--  1 dan users  119 Jan 12 09:42 config
-rw-r--r--  1 dan users   73 Jan 12 09:42 description
-rw-r--r--  1 dan users   23 Jan 12 09:42 HEAD
drwxr-xr-x  2 dan users 4096 Jan 12 09:42 hooks
drwxr-xr-x  2 dan users 4096 Jan 12 09:42 info
drwxr-xr-x 13 dan users 4096 Jan 12 09:42 objects
-rw-r--r--  1 dan users   85 Jan 12 09:42 packed-refs
drwxr-xr-x  4 dan users 4096 Jan 12 09:42 refs
\end{lstlisting}
If you don't see this directory structure then STOP, you're in the wrong directory.

Now run the following
\begin{lstlisting}
$ cp /usr/share/doc/git/contrib/hooks/post-receive-email hooks/post-receive
$ chmod a+x hooks/post-receive
$ git config --local hooks.mailinglist "yourgroupemail@doc.ic.ac.uk"
$ git config --local hooks.emailprefix "[SUBJECT PREFIX]"
$ echo "Project Name" > description
\end{lstlisting}
Where \texttt{yourgroupemail@doc.ic.ac.uk} is the e-mail address for messages to be sent to, \texttt{[SUBJECT PREFIX]} is the subject prefix to put on the e-mail messages (useful for filtering) and \texttt{Project Name} is the name of your project (try to keep it short as it will appear in the e-mails.

If you look at the file \texttt{hooks/post-receive} it describes some of its other options. If you want to change some of the static text in the e-mail take a look at the \texttt{generate\_email\_header()} and \texttt{generate\_email\_footer()} functions.

\subsection{Public centralised repository on GitHub}

\section{A few tips}
\begin{itemize}
	\item This isn't a \texttt{git} tip but it's useful. You might find it hard to remember the path to your group folder (I know I do). To make life easier make a symbolic link to it in your home directory.
\begin{lstlisting}
#Go into your home directory
$ cd ~
# Create a symbolic link to "/vol/project/2011/530/group-name/" in your home directory called "group-folder"
$ ln -s /vol/project/2011/530/group-name/  group-folder
\end{lstlisting}


	\item If \texttt{shell1} goes down as it does from time to time but one of the other three are running you can add a ``remote'' to use one of the other shells to access your repository through a different machine. Note lines beginning with \texttt{\#} are comments and should not be typed in.
	\begin{lstlisting}
# Listing current remotes
$ git remote -v
origin  /vol/project/2011/530/g1153011/repo-name (fetch)                                                                                                                                                                                                                       
origin  /vol/project/2011/530/g1153011/repo-name (push) 
#Add remote
git remote add origin2 username@shell2.doc.ic.ac.uk:/vol/project/2011/530/g1153011/repo-name
#fetch from origin2's master branch and merge into current brach
git pull origin2 master
	\end{lstlisting}
	
	\item
	When working on features it is usually more sensible to make a branch off the ``master'' per feature so you can work on different parts independently.
\begin{lstlisting}
$ git checkout -b feature1
#then later after changes have been made in feature1
$ git checkout master
Switched to branch 'master'
$ git merge feature1
Updating 33947d6..8969c25
Fast-forward
 readme |    2 +-
 1 files changed, 1 insertions(+), 1 deletions(-)
\end{lstlisting}

	\item If you're working on a branch but then need to look at another but your changes aren't ready to be committed you can ``stash'' them away for committing latter. See \url{http://book.git-scm.com/4_stashing.html} and \texttt{man git stash}. 

\end{itemize}


\section{That's it}
Hope this was useful. Enjoy!
\end{document}
